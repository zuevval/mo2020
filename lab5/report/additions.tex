%XeLaTeX+makeIndex+BibTeX OR LuaLaTeX
\documentclass[main.tex]{subfiles}

\begin{document}
	\section{Ответы на вопросы}
	\emph{Q:} Поясните выбор решения задачи линейного программирования. \\
	\emph{A:} Для нахождения начального приближения используется метод перебора, потому что на первой итерации в ограничения входят 6 плоскостей, т. о. двойственная задача поставлена в $\mathds{R}^6$ и всего нужно перебрать не более $2^5=32$ различных векторов, что занимает немного времени на современном компьютере. Далее для решения задачи используется симплекс-метод, что рационально сделать с учётом увеличения размерности и возможности использования решения двойственной задачи, полученного на предыдущей итерации, для построения начального приближения.\\
	\emph{Q:} Если делать перебор, то зачем решать двойственную задачу?\\
	\emph{A:} Специфика прямой задачи такова, что построение двойственной сразу даёт задачу в канонической форме. Если решать прямую задачу, её надо сперва привести к канонической форме.
\end{document}
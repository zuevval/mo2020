%XeLaTeX+makeIndex+BibTeX OR LuaLaTeX

\documentclass[a4paper,12pt]{article} %14pt - extarticle
\usepackage[utf8]{inputenc} %русский язык, не менять
\usepackage[T2A, T1]{fontenc} %русский язык, не менять
\usepackage[english, russian]{babel} %русский язык, не менять
\usepackage{fontspec} %различные шрифты
\setmainfont{Times New Roman}
\defaultfontfeatures{Ligatures={TeX},Renderer=Basic}
\usepackage{pdfpages} % insert pdf pages, e.g. scans
\usepackage{hyperref} %гиперссылки
\hypersetup{pdfstartview=FitH,  linkcolor=blue, urlcolor=blue, colorlinks=true} %гиперссылки
\usepackage{subfiles}%включение тех-текста
\usepackage{pdfpages}
\usepackage{graphicx} %изображения
\usepackage{float}%картинки где угодно
\usepackage{textcomp}
\usepackage{fancyvrb} %fancy verbatim for \VerbatimInput
\usepackage{dsfont}%мат. символы
\usepackage{amssymb, amsmath} %common math symbols
\newcommand{\norm}[1]{\left\lVert#1\right\rVert}
\usepackage[linesnumbered, ruled]{algorithm2e}%pseudocode
\usepackage{listings} %code formatting
\lstset{language=python,
	%keywordstyle=\color{blue},
	%commentstyle=\color{},
	%stringstyle=\color{red},
	tabsize=1,
	breaklines=true,
	columns=fullflexible,
	%numbers=left,
	escapechar=@
	%morekeywords={numpy, np}
}

\textwidth=16cm
\oddsidemargin=0cm %поля (отступ слева)
\topmargin=-1cm %поля (отступ сверху)
\textheight=23cm

\newcommand{\myPictWidth}{.8\textwidth}
\graphicspath{{./img/}}

\begin{document}
	\subfile{titlepage}
	\tableofcontents \newpage
	\subfile{body} \newpage
	
	\begin{thebibliography}{99}
		\bibitem{petuh} Петухов Л. В. Методы оптимизации. Задачи выпуклого программирования: учеб. пособие / Л. В. Петухов, Г. А. Серёгин, Е. А. Родионова. -- СПб.: Изд-во Политехн. ун-та, 2014. -- 99 с.
		\bibitem{boldirev} Методы оптимизации. Математическое программирование: учеб. пособие. / Ю.Я. Болдырев, Е.А. Родионова; С.-Петерб. гос. техн. ун-т. - СПб.: Изд-во СПбГТУ, 1999. -- 81 с.: ил.
		% \bibitem{karmanov} Карманов, В.Г. Математическое программирование: Учеб. пособие. -- 6-е изд., испр. -- М.: ФИЗМАТЛИТ, 2008. -- 264 с. % yet unused
		% \bibitem{galeev} Галеев, Э. М., Тихомиров, В. М. Оптимизация: теория, примеры, задачи. -- М.: Эдиториал УРСС, 2000. -- 320 с. % yet unused
	\end{thebibliography}
	
\end{document}
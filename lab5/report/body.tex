\documentclass[main.tex]{subfiles}

\begin{document}
	
\section{Постановка задачи}
Поставлена задача условной оптимизации следующего вида:
\begin{gather*}
\overline{\phi_0}(x) \rightarrow min_{x} \\
x \in \overline{\varOmega} := \{x|\phi_1(x) \le 0, \phi_2(x) \le 0\} 
\end{gather*}
где $x \in \mathds{R}^2$, $\phi_1, \phi_2$ -- выпуклые функции, подобранные так, чтобы множество $\overline{\varOmega}$ было замкнутым; $\overline{\phi_0}$ -- нелинейная функция цели.\\
\begin{enumerate}
	\item Привести задачу к виду, пригодному для применения метода отсекающей гиперплоскости:
	\begin{gather*}
	\phi_0(x) = (c, x) \rightarrow min_{x} \\
	x \in \varOmega := \{x|\phi(x) \le 0\} 
	\end{gather*}
	где $\phi$ -- выпуклая функция.
	\item Построить замкнутый многогранник $S:=\{x|Ax\le b\}$ с условием $\varOmega \subset S$, необходимый для начального этапа метода отсекающей гиперплоскости.
	\item Решить задачу методом отсекающей гиперплоскости, на каждом шаге решая проблему линейной минимизации следующим образом: проблему привести к двойственному виду, применить симплекс-метод и восстановить решение прямой по решению двойственной.
\end{enumerate}
\section{Обоснование применимости методов}
\section{Описание алгоритмов}
\section{Результаты решения задачи}
\section{Оценка достоверности полученных результатов}
\end{document}
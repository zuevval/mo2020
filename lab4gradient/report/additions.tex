\documentclass[main.tex]{subfiles}

\begin{document}
	\section{Ответы на вопросы}
	\emph{Вопрос.} Почему выбран квадрат нормы градиента функции цели в условии окончания? \\
	\emph{Ответ.} Для выхода из итерационного процесса могут быть использованы два равносильных условия:
	\begin{equation}\label{eq:stopping1}
		\norm{\nabla f} \le \varepsilon
	\end{equation}
	\begin{equation}\label{eq:stopping2}
		\norm{\nabla f}^2 \le \overline{\varepsilon}, \overline{\varepsilon} = \varepsilon^2 
	\end{equation}
	Если выбрана первая или бесконечная норма, т. е. $\norm{\vec{x}}_1 := \sum_{k=1}^{n}x_k$ или $\norm{\vec{x}}_{\infty} := \max x_k$, для упрощения вычислений легче пользоваться критерием (\ref{eq:stopping1}). Если вторая норма $\norm{\vec{x}}_2 := \sqrt{\sum_{k=1}^{n}x_k^2}$, то с точки зрения вычислений проще пользоваться критерием (\ref{eq:stopping2}), что и осуществлено в данной работе.\\
	\\
	\emph{Вопрос.} Поясните подробнее вычислительный эксперимент для получения оценок $m, M: m\norm{y}^2 \le y^{T}H(x)y \le M\norm{y}^2 \forall y \in \mathds{R}^n \forall x \in S$, где $S \subset \mathds{R}^n$ -- множество, на котором ищем $m, M$ \\
	\emph{Ответ.} Заметим, что 
	$$m = \inf_{\norm{y=1}=1, x \in S} \norm{y^{T}H(x)y}$$
	$$M = \sup_{\norm{y=1}=1, x \in S} \norm{y^{T}H(x)y}$$
	Множество $S$ удобно задать прямоугольником. Тогда алгоритм численного поиска таков.\\
	Рассматриваем двумерный случай. Пусть $h_{11}(\vec{x}), h_{12}(\vec{x}) = h_{21}(\vec{x}), h_{22}(\vec{x})$ -- коэффициенты матрицы Гессе, $S=[\vec{a};\vec{b}]$. Зададим малые числа $\delta_1, \delta_2, \delta_\phi$\\
	\begin{algorithm}[H]\label{gradient}
		$m := \infty$\;
		$M := -\infty$\;
		\For{$x_1=a_1; x_1 \le b_1; x_1 += \delta_1$}{
			\For{$x_2=a_2; x_2 \le b_2; x_2 += \delta_2$}{
				\For{$\phi=0; \phi \le 2*\pi; \phi += \delta_\phi$}{
					$y := y=\begin{pmatrix} \cos(\phi)\\ \sin(\phi)\end{pmatrix}$\;
					$H := y^T H\left(\begin{pmatrix}x_1\\x_2\end{pmatrix}\right) y$\;
					\If{$m > H$}{$m=H$\;}
					\If{$M < H$}{$M=H$\;}
				}
			}
		}
	\end{algorithm}
	Поскольку производные $h_{11}, h_{12}, h_{22}$ ограниченны, то с измельчением частот дискретизации $\delta_1, \delta_2, \delta_\phi$ точность вычисления $m$, $M$ возрастает.
\end{document}
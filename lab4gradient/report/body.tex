\documentclass[main.tex]{subfiles}

\begin{document}
	
\section{Постановка задачи}
Поставлена проблема двумерной минимизации:
\begin{equation}\label{eq:problem}
f(x)=x_1^2+x_2^2+cos(x_1+3x_2)-x_1+2x_2
\end{equation}
Необходимо
\begin{enumerate}
	\item Решить задачу (\ref{eq:problem}) методом наискорейшего спуска
	\item Доказать сходимость метода наискорейшего спуска применительно к данной функции
	\item Решить ту же задачу методом Ньютона второго порядка.
\end{enumerate}

\section{Обоснование применимости методов}
\section{Описание алгоритмов}
\subfile{gradient.tex}
\subsection{Метод Ньютона}

\section{Результаты решения задачи}
\section{Оценка достоверности результата}

\begin{thebibliography}{99}
	\bibitem{galeev} Галеев, Э. М., Тихомиров, В. М. Оптимизация: теория, примеры, задачи. -- М.: Эдиториал УРСС, 2000. -- 320 с.
\end{thebibliography}

\end{document}